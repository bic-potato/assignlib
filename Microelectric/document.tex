\documentclass[UTF8]{ctexart}
\title{第4次作业}
\author{左熙辰}
\begin{document}
\maketitle
\section{描述二极管的工作机理}
当一块P型半导体与一块N型半导体连接时,在两者交界处会发生空穴与电子的复合,最后产生耗尽区,形成自建场和自建电势。自建电势和载流子的动力势相抵消,使载流子停止流动。

当外接电源正极与p区相连时,外加电场与自建电场相反,当外加电场足以抵消自建电势时,载流子顺浓度梯度流动,半导体通电。

当外接电源正极与n区相连时,外加电场与自建电场相同,载流子所受阻力较大,仅有少数载流子做漂移运动,电阻极大。
\section{讨论PNP双极晶体管的工作原理}
PNP双极晶体管由两个距离很近的NP节构成。其中N区长度极短,称为基区:一段的P区添加正向小电压,称为发射极,另一端加反向大电压,称收集极。

发射极电子射入基区时一部分电子未与空穴复合就扩散至收集结附近并被收集区的大电压抽吸至收集区,形成较大的反向电流。


\end{document}