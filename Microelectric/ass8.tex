\documentclass[utf8]{ctexart}
\title{Assignment 8}
\author{左熙辰-2000012103}
\date{}
\begin{document}
	\maketitle
	\section{材料膜的生长}
		\subsection{分类}
			\begin{itemize}
				\item{化学气相淀积}
				\item{物理气相淀积}
				\item{SiO$_{2}$的热氧化}
			\end{itemize}
		\subsection{作用}
			在硅片上生长材料膜
	\section{光刻}
		\subsection{分类}
			\begin{itemize}
				\item{接触式}
				\item{接近式}
				\item{投影式}
			\end{itemize}
		\subsection{作用}
		将掩模版上的图形转移到硅片表面材料层上。
	\section{刻蚀}
		\subsection{分类}
			\begin{itemize}
				\item{湿法刻蚀}
				\item{干法刻蚀}
					\subitem{溅射与离子束刻蚀}
					\subitem{等离子刻蚀}
					\subitem{反应离子刻蚀}
			\end{itemize}
		\subsection{作用}	
		在基片上形成真正的图形和三维结构
	\section{扩散与离子注入}
		\subsection{分类}
			\begin{itemize}
				\item{固态源扩散}
				\item{气态源扩散}
				\item{液态源扩散}
				\item{离子注入}
			\end{itemize}
		\subsection{作用}
			对高纯硅进行掺杂
	\section{接触与互连}
		\subsection{分类}
			\begin{itemize}
				\item{CMP}
				\item{Cu互连的大马士革工艺}
				\item{Salicide结构}
			\end{itemize}
		\subsection{作用}
			将硅片上各个电学器件相互联通
\end{document}