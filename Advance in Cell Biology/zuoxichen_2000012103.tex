\documentclass[utf8,10.5pt]{ctexart}
\usepackage{amsrefs}
\usepackage{amsmath}
\usepackage{amssymb}
\usepackage{geometry}
\geometry{scale=0.85}
\author{}
\date{}
\title{免疫与衰老}
\begin{document}
    \maketitle
    免疫学,是生命科学中最为活跃的学科之一,是和人类生活联系最密切的学科之一,也是人类解决传染病(如COVID-19)和相当一部分非传染病(如肿瘤)最有效的解决办法。
    同时,免疫的失调也会牵涉一系列自身疾病,如AIDS,系统性红斑狼疮等。同时,由于免疫系统的复杂性,人类目前对此知之甚少,有大量的领域没有建立一个完整的体系,仍有许多地方需要人类探索。
    免疫同时与医学、生物医药行业紧密联系,为理论转化为实际生产提供了一个相当良好的平台。

    目前,我们意识到,人类免疫系统老化是人类衰老过程中一个十分重要而人类又对此知之甚少的过程,甚至对其的定义都较为混乱。
    近些年人类倾向于将只有年轻人和老年人之间的免疫参数差异,以某种明确的方式与有害的健康结果和/或受损的生存期望相关的指标归为免疫系统老化指标。
    然而我们对这些指标知之甚少,对所有可能的因子的鉴定需要确定下列问题:
    (1)什么尽管在人类群体中的普遍性尚未得到证实但确实似乎与年龄相关;
    (2)什么显然是随着年龄而发生的一系列免疫系统典型变化的一部分;
    (3)这些变化中的哪一个子集加速而不是减缓衰老;以及
    (4)所有可能因人群而异的加速衰老的变化。

    在各种可能的衰老因子中,CD8$^+$细胞一直是一个巨大的谜团。早在2000年,科学家们就发现幼CD8$^+$ T细胞在血液中的数目减少(Fagnoni等, 2000)
    ,这是符合人们预期的,但与预期不符的是,CD8$^+$记忆细胞在老年人血液中积累的报道并不普遍。与此同时,很明显,在老年人中看到的晚期记忆细胞的积累是由人类疱疹病毒5(HHV5)引起的而并不是由于衰老引起的。
    并且,虽然树突状细胞(DC)和中性粒细胞存在着一定的年龄相关差异(Stervbo等,2015a;Stervbo等,2015b),但远没有CD8$^+$ T细胞明显.但是由于免疫细胞大多存在于组织而非血液中,
    在未来的研究中,一系列关于细胞亚群在各组织中的分布随年龄增大变化的情况需要被测定。

    在SARS-CoV-2大流行中老年人的死亡率相对最高,故研究免疫衰老则显得十分必要。由于幼CD8$^+$ T细胞受年龄影响最大,对于新的抗原暴露,潜在的“谱系漏洞”是最大的问题。
    而对于此,尽管在动物实验中得到了证实,但在人类中的数据仍然缺乏。
    与适应性免疫相反的是,先天免疫在脊椎动物和无脊椎动物中都是保守的(Muller等人,2013年),基本上保持了功能,甚至随着年龄的增长变得过于活跃。
    
    尽管人们意识到, 为了严格定义免疫衰老,只有那些已经被证明与有害的临床结果相关的差异(例如,死亡率、虚弱、对疫苗的不良反应等)应该包括在内并在大多数已发表的研究记录了老年人和年轻人之间的差异,但没有将这些差异与可测量的不良临床条件或死亡率联系起来。此外,通过纵向研究,大多数报告的差异还没有被正式证明是由于年龄增长而产生的变化。

    尽管在横向研究中,一些免疫参数清楚地证明了年轻和老年人群之间的多重差异,并且动物研究以及一些更有限的人类纵向研究表明,这些差异中的许多确实可能是个体内与年龄和环境相关的变化。某些免疫特征被确定为随年龄发生明显变化,可能与重要的身体状态有关。但其中许多可能只在接受评估的人群中提供信息,这些是否作为老化过程本身的反映还有待商榷(Waaijer等人,2019年)。而寻找真正普遍的与年龄相关的免疫标记物变化的工作仍在进行中。
    \end{document}