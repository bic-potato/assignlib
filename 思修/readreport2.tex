\documentclass[utf8]{ctexart}
\usepackage{geometry}
\title{读书报告-马克思《资本论》第三卷第23章}
\author{左熙辰-2000012103}
\date{}
\geometry{a4paper,left=2.70cm,right=2.5cm,top=3.8cm,bottom=3.6cm}
\begin{document}
	\maketitle
	《资本论》中译本是由中共中央马克思、恩格斯、列宁、斯大林著作编译局翻译,由人民出版社于2004年出版的一篇巨作。作者为卡尔·海因里希·马克思。
	
	马克思,德国的思想家、政治学家、哲学家、经济学家、革命理论家、历史学家和社会学家,第一国际的组织者和领导者,马克思主义政党的缔造者之一,马克思主义的创始人之一,全世界无产阶级和劳动人民的最伟大的革命导师,是国际共产主义运动的开创者。早年在德国求学并获得耶拿大学哲学博士。毕业后创办了报纸《莱茵报》并任主编,因为“林木盗窃问题”而遭普鲁士反动政府迫害。在此期间,马克思结识了一生的革命挚友-恩格斯。
	
	随着资本主义生产方式在欧洲社会的迅猛发展,资本主义社会形态下的固有矛盾愈发明显地暴露出来。无产阶级反对资产阶级的斗争日益尖锐,呈现复杂化的趋势。在这种背景下,马克思为了给无产阶级提供强大的理论武器,开始着手研究政治经济学,并最终产生了《资本论》。
	
	《资本论》第一卷通过对直接生产过程的分析,揭示了资本主义的一般基础(商品经济)、剩余价值的秘密、资本的本质、资本主义的基本矛盾及其发展的历史趋势,因而从根本的层次上阐明了资本主义经济中的最主要、最基本的问题,是全书的基础。第二卷考察的是广义的资本流通过程;即除了直接生产过程外,加入了交换过程。这一卷主要分析单个资本的再生产(资本的循环和周转)和社会总资本的再生产,揭示资本主义的微观经济和宏观经济的运行过程。第三卷是对资本运动总过程的分析。补充回答了前两卷暂时存而未论的问题,并且前面做过的许多分析(如价值、货币等)进行了更加深入、更加丰富、更加具体的论证。
	
	其中,《资本论》第三卷第三篇主要讲述了利润率趋向下降的原因(第13章)、对利润率趋向下降起反作用的各种原因(第14章)并仔细分析了规律本身的内部矛盾(第15章)如生产扩大和价值增值之间的冲突、人口过剩时的资本过剩。
	
	文章(第3卷第14章)首先阐释了剩余价值率和利润率的概念和区别,并指出,在剩余价值率不变甚至升高的情况下利润率会下降。利润率的下降是资本主义生产方式的必然结果。资本家为了追逐超额利润以及在竞争中取得优势,不断地竞相改进生产技术,在愈来愈多的生产环节中用机器来代替手工劳动,这就必然要引起各个生产部门资本有机构成不断提高,从而引起社会资本的平均有机构成的提高,使平均利润率趋向下降。但利润率的下降伴随着利润量的增加。第14章阐述了对利润率趋向下降起反作用的各种原因:I.劳动剥削程度的提高、II.工资被压低到劳动力的价值以下、III.不变资本各要素变得便宜、IV.相对过剩人口、V.对外贸易。第15章进一步阐释了规律背后的深层次的矛盾:发展生产力的趋势和保存现有资本价值和最大程度增值价值之间的矛盾。
	
	当代资本主义的发展并未超过《资本论》的分析。其指出:不是财富生产的太多,而是资本主义的、对立的形式上的财富、周期地生产太多了。资本家为了攫取高额利润,不惜一切代价扩大生产并剥削工人,甚至刻意哄抬价格,而无产阶级的购买能力并未随之增加,导致贫富差距进一步扩大。
	
	
	对于《资本论》的研究又可以为当下有中国特色社会主义的建设指明价值方向。虽然改革开放四十周年以来,我国的经济水平、综合国力有着质的飞跃,但是,市场经济所存在的痼疾在资本主义社会存在,在社会主义社会也不会自发消失。伴随着经济的高速增长,我国同样出现了一系列问题,如收入分配差距不断扩大,贫富两极分化凸显; 城乡差距进一步拉大,农民工问题、农村留守儿童问题; 各种生态环境污染事件; 产能过剩、投资收益率下降、房价居高不下、人与人之间金钱利益至上,社会道德滑坡等。《资本论》所提出的是“发展为了谁,发展依靠谁,发展成果由谁享有”的严肃命题需要我们好好反思。如果只是一味地追求财富的累积与经济发展的速度,这样的发展最终会导致社会的永久性撕裂,即使创造出足够的财富,也并不能造福大多数人而只能会成为少数人随意挥霍的资本。
	
	《资本论》同时也指出了中国特色社会主义制度的必要性:即在生产力未得到大发展,劳动力未得到大解放之前,需要借助市场经济体制加快生产力发展,同时也向我们展现了社会主义代替资本主义这个过程的长期曲折性、艰巨复杂性。需要我们坚持不断的去奋斗。
	
	资本主义生产的真正限制在于资本自身:资本的增值表现为生产的起点和终点;生产只是为资本的增值而服务,而不是为了生产者的社会过程扩大的手段。以对广大生产者的被剥夺和贫穷化为基础的保存和增值,只能在一定的范围内运动。资本为了突破这个界限,这种矛盾最终的表现就是周期性的资本主义经济大危机.所以资本主义的崩溃是必然的结果。
	
	近日蚂蚁金服事件充分证明了资本为扩大生产的不择手段与贪婪无厌。利用法律漏洞假借高科技产品在科创版上市,企图突破政府监管,疯狂试探政府底线。随着原有体系利润率的降低,企图利用无抵押借贷,以全国的经济为其背书,这种行为必须禁止。
	
\end{document}