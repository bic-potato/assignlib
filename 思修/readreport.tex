\documentclass[UTF8]{ctexart}
\usepackage[namelimits]{amsmath} %数学公式
\usepackage{amssymb}             %数学公式
                                 %数学字体
\usepackage{mathrsfs}            %数学花体
\usepackage{geometry}
\geometry{a4paper,left=2.70cm,right=2.5cm,top=3.8cm,bottom=3.6cm}
\title{读书报告-共产党宣言}
\author{左熙辰-2000012103}
\begin{document}
    \maketitle
    《共产党宣言》(又译《共产主义宣言》,以下简称宣言)是马克思和恩格斯为共产主义者同盟起草的纲领,是马克思主义诞生的重要标志。由马克思和恩格斯执笔写成 。1848年2月21日在伦敦第一次以单行本问世。2月24日,《共产党宣言》正式出版。1920年8月,上海的社会主义研究社出版了陈望道从日文本并且参照英文本翻译的《共产党宣言》全译本。现行中译本由中共中央马克思、恩格斯、列宁、斯大林著作编译局译;由人民出版社出版的单行本。

    《宣言》首次提出了科学社会主义理论,并指出共产主义运动已成为不可抗拒的历史潮流。《共产党宣言》运用辩证唯物主义和历史唯物主义分析生产力与生产关系、经济基础与上层建筑的矛盾,分析阶级和阶级斗争,特别是资本主义社会阶级斗争的产生、发展过程,论证资本主义必然灭亡和社会主义必然胜利的客观规律,作为资本主义掘墓人的无产阶级肩负的世界历史使命。《宣言》公开宣布必须用暴力革命推翻资产阶级统治,以无产阶级专政代替资本主义专政,进而消灭阶级存在的条件,最后代替存在着阶级和阶级对立的资产阶级社会的,是一个每个人的自由发展是一切人的自由发展的条件的联合体,即共产主义社会。

    《宣言》引言说明了《宣言》产生的历史背景和任务;第一章论述了马克思主义的阶级斗争学说;《宣言》第二章《无产者和共产党人》,说明了无产阶级政党的性质、特点、目的和任务,以及共产党的理论和纲领。宣言》第三章《社会主义的和共产主义的文献》,批判了当时流行的各种假社会主义,分析了各种假社会主义流派产生的社会历史条件,并揭露了它们的阶级实质。分为1、反动的社会主义。2、保守的或资产阶级的社会主义。3、批判的空想的社会主义和共产主义。
    《宣言》第四章《共产党人对各种反对党派的态度》论述了共产党人革命斗争的思想策略。
    
    《宣言》向所有共产党人指明了出路:即只有通过暴力革命推翻现存的制度才能走向共产主义。

    《宣言》中的深邃的思想依旧指引着我们,其中的血淋淋的事实也一再地在现实中发生。
    
    “当厂主对工人的剥削告一段落,工人领到了用现钱支付的工资的时候,马上就有资产阶级中的另一部分人——房东、小店主、当铺老板等等向他们扑来。”这句写于19世纪的话,到现在还依旧真实的刺眼。“当老板对员工的剥削告一段落,工人领到了工资的时候,马上就有马上就有资产阶级中的另一部分人——房东、淘宝、花呗等等向他们扑来。”资本家对无产阶级的剥削从未停止,只是披着的外皮“与时俱进”而已。高额的房租、从11月1日开始的“双十一”、诱导超前消费的花呗...这一切,都是以“人民”资本家马云先生为代表的资本家老爷们用来吸干无产者身上最后一滴血的方式而已。

    有人会责备我们:“说我们要消灭个人挣得的、自己劳动得来的财产,要消灭构成个人的一切自由、活动和独立的基础的财产。”好一个劳动得来的、自己挣得的、自己赚来的财产!难道你们的雇佣劳动真的会给无产者带来财产吗?没有。无产者的劳动带来的是资本,“即剥削雇佣劳动的财产”,而雇佣工人靠自己的劳动所占有的,只够“维持他们的生命的再生产”,绝大部分人在一线城市享受着看似高额的工资,可也仅仅能勉强偿还房贷而已。
    
    我们要消灭私有制,有些人就指责起来:“人家自己的‘努力’,凭什么不是自己的。”他们的努力是什么?努力地让整个社会2\%的人拥有99\%的财富吗?这种私有制之所以存在,正是因为私有财产对于绝大部分人已经不存在。在现存社会里,私有财产对99\%的成员来说已经被消灭了;你们责备我们,是说我们要消灭那种以社会上的绝大多数人没有财产为必要条件的所有制。而现在无产阶级的一部分人却还要跪着去舔那些人脚上的尘土!

    我们要消灭生而不平等,你们跳出来说“人家几代人的努力,凭什么让你一代人超越?”你们想传几代?传出个王位?传出一个绝大部分人永远也爬不到别人起跑线的社会?我们当时为什么要打倒地主?我们当时为什么要打倒资本家?为什么我们要去重蹈我们之前亲手推翻的错误?真就王思聪的狗比人贵呗?

    我们要消灭的私有制,不是供直接维持生命用的劳动产品的个人占有,这种占有并不会留下任何剩余的东西使人们有可能支配别人的劳动。我们要消灭的是个人占有资本的性质,在这种占有中,无产阶级仅仅为资本家的利益活着,只有资本家在需要他活着的时候才活着。

    我们要消灭的不平等,不是高考招生形式上的公平,这种“公平”百害无一利,而是教育资源上的不均匀分配,是即使你为之付出,你也与一些人之间有着不可逾越的鸿沟。

    你们追求的教育改革,是要把教育改革成“资产者唯恐失去的那种教育”,是对绝大多数人来说是把人训练成一辈子只能被资本家剥削的人的教育,
    
    在资本主义社会中,获者不劳,劳者不获,才会产生“摸鱼”文化,人们厌恶自己的工作,毋宁说,厌恶自己的劳动被剥削。

    “民族内部的阶级对立一消失,民族之间的敌对关系就会随之消失。”所有的矛盾都是阶级矛盾。女权问题本身是一种阶级矛盾的具化。由于自身的生理特点,决定了在资本主义逐利的情况,由于女性参与的社会劳动较男性更少,自然女性地位更低。“妇女解放,是使妇女与男人享受平等的社会权利,成为全面而自由发展的人。要做到这一点,须消灭私有制。”
    而现在所谓的田园女拳,恶意挑动性别对立,激化矛盾,对主要矛盾视而不见,看似为女性发生,实则只是资本家分化无产阶级力量的方式。而美国红脖子的出现,也只是美国资本家的诡计罢了。

    为什么近些年资本家的争议越来越大?
    
    “如果人民从具体的某位资本家的议题逐渐聚焦到资本家存在本身的合理性,就代表...”

    “代表什么?”

    \textbf{“他回来了。”}
    
    “他是谁?”

    “他是马克思,是恩格斯,是列宁,是毛泽东,是资本家最害怕的幽灵。”

    “他的名字,叫共产主义。”

    让统治阶级在共产主义革命面前发抖吧。无产者在这个革命中失去的只是锁链。他们获得的将是整个世界。
    
    全世界无产者,联合起来!
\end{document}