\documentclass[utf8]{ctexart}
\usepackage{geometry}
\title{读书报告-马克思《资本论》第三卷第23章}
\author{左熙辰-2000012103}
\date{}
\geometry{a4paper,left=2.70cm,right=2.5cm,top=3.8cm,bottom=3.6cm}
\begin{document}
	\maketitle
	《资本论》中译本是由中共中央马克思、恩格斯、列宁、斯大林著作编译局翻译,由人民出版社于2004年出版的一篇巨作。作者为卡尔·海因里希·马克思。
	
	马克思,德国的思想家、政治学家、哲学家、经济学家、革命理论家、历史学家和社会学家,第一国际的组织者和领导者,马克思主义政党的缔造者之一,马克思主义的创始人之一,全世界无产阶级和劳动人民的最伟大的革命导师,是国际共产主义运动的开创者。早年在德国求学并获得耶拿大学哲学博士。毕业后创办了报纸《莱茵报》并任主编,因为“林木盗窃问题”而遭普鲁士反动政府迫害。在此期间,马克思结识了一生的革命挚友-恩格斯。
	
	
	
	随着资本主义生产方式在欧洲社会的迅猛发展,资本主义社会形态下的固有矛盾愈发明显地暴露出来。无产阶级反对资产阶级的斗争日益尖锐,呈现复杂化的趋势。在这种背景下,马克思为了给无产阶级提供强大的理论武器,开始着手研究政治经济学,并最终产生了《资本论》。
	
	《资本论》第一卷通过对直接生产过程的分析,揭示了资本主义的一般基础(商品经济)、剩余价值的秘密、资本的本质、资本主义的基本矛盾及其发展的历史趋势,因而从根本的层次上阐明了资本主义经济中的最主要、最基本的问题,是全书的基础。第二卷考察的是广义的资本流通过程;即除了直接生产过程外,加入了交换过程。这一卷主要分析单个资本的再生产(资本的循环和周转)和社会总资本的再生产,揭示资本主义的微观经济和宏观经济的运行过程。第三卷是对资本运动总过程的分析。补充回答了前两卷暂时存而未论的问题,并且前面做过的许多分析(如价值、货币等)进行了更加深入、更加丰富、更加具体的论证。
	
	其中,《资本论》第三卷第23章主要讲述了利息、企业主收入与剩余价值之间的关系。即利息与企业主收入是剩余价值的两种表现形式,其两者存在质的区别。利息是由生息资本转化形成的剩余价值,利息是依托于实业的存在,该章同时深入分析了无产阶级中的一类特殊群体的存在,他们既和绝大多数无产阶级有着共同的利益,也存在这不可调和的矛盾。该章最后分析了一种新出现的新的剥削类型:董事与监工。
	
	文章(第三卷第23章)首先分析了货币资本与产业资本的区别与联系以及两者的不可分割性,随后引出了企业主利息的概念,并详细比较了企业主收入与利息之间在量与质上的区别,再通过企业主利息引出了监督工资的概念,并指出了资本家与无产者之间矛盾的必然性。最后对企业主收入进行了更深入的描述,并引出了监督劳动的概念,对将监督劳动而产生的对劳动力的奴役进行美化的说法进行了批判,并指明了产业资本家相对货币资本家属于劳动者,而其劳动的工资是建立在对其占有的劳动的剥削程度的深刻内涵,最后,引出董事与监工的概念,并指明这是由企业主收入和监督收入相混淆后的一种欺诈结果。
	
	当代资本主义的发展并未超过《资本论》的分析。在新自由主义的主导下,发达国家资本大举流入金融和虚拟经济,整个西方经济的实体部分下降为不到 30\% ,虚拟部分上升到 70\% 以上,其中大部分是金融。在此期间,资本尤其是大资本享受超国民待遇,而工人福利被不断缩减,资本的收益率的增长远远超过工人工资的增长; 工人工资不足导致有效需求不足、有效需求不足又进一步导致实体经济萎缩,实体经济的萎缩最终又导致工人福利的进一步削减,最终引发了一场旷日持久的金融和经济危机,造成了一系列社会问题。这充分体现了货币资本和产业资本的不可分割性。
	
	对于《资本论》的研究又可以为当下有中国特色社会主义的建设指明价值方向。虽然改革开放四十周年以来,我国的经济水平、综合国力有着质的飞跃,但是,市场经济所存在的痼疾在资本主义社会存在,在社会主义社会也不会自发消失。伴随着经济的高速增长,我国同样出现了一系列问题,如收入分配差距不断扩大,贫富两极分化凸显; 城乡差距进一步拉大,农民工问题、农村留守儿童问题; 各种生态环境污染事件; 产能过剩、投资收益率下降、房价居高不下、资本脱实向虚; 人与人之间金钱利益至上,社会道德滑坡等。《资本论》所提出的是“发展为了谁,发展依靠谁,发展成果由谁享有”的严肃命题需要我们好好反思。如果只是一味地追求财富的累积与经济发展的速度,这样的发展最终会导致社会的永久性撕裂。
	
	《资本论》同时也指出了中国特色社会主义制度的必要性:即在生产力未得到大发展,劳动力未得到大解放之前,需要借助市场经济体制加快生产力发展,同时也向我们展现了社会主义代替资本主义这个过程的长期曲折性、艰巨复杂性。需要我们坚持不断的去奋斗。
	
	在当下的美国,看起来资产阶级统治根深蒂固,实则是无产阶级被资本家们分化、愚化,华尔街故意在社会中传播反智的不良风气,在无产阶级中传播种族主义的思想,教唆他们试图通过剥削别的民族来不劳而获,从而达到巩固美国资本家们统治的目的。并且,由于长时间西方媒体的恶意抹黑,美国的无产阶级对共产党有着深深的恐惧情绪。并且由于华尔街的收买与笼络,一部分人堕落为流氓无产阶级,甘愿当华尔街的走狗。
	
	
\end{document}