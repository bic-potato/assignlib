\documentclass[utf8]{ctexart}
\title{讲稿}
\author{}
\date{}
\usepackage[namelimits]{amsmath} %数学公式
\usepackage{amssymb}             %数学公式
\usepackage{mathrsfs}            %数学花体
\usepackage{geometry}
\geometry{a4paper,scale=0.85}
\begin{document}
	\maketitle
	\section{规律本身}
	\subsection{剩余价值率与利润率}
	我们先来考察以下内容:
	
	假设100个人的工资100刀,产生了200刀的商品并全部卖出,总共产生的剩余价值率就是:$$\frac{100}{100}=100\%$$而利润率却还要加入生产这批产品的总投资,所以,在剩余价值率不变时,投资越多,所产生的利润率变低。
	
	好,大概你是一脸懵逼的,那我们再来看看麻辣香锅的例子:假设我们是资本主义道路,李健老师是大资本家,现在我们的无偿劳动归他所用,他要用这笔钱生产更多的麻辣香锅,首先,他用了这笔钱买了更多的原材料制成了香锅,现在,香锅原料的价值就转移到制成的香锅上了,但是这一部分没有发生改变,所以是不变资本,而真正的增值的资本是在转移的过程中所附加的无差别劳动,所以我们自然可以得出这样的结论:在固定剩余价值率的情况下,利润率会随规模的扩大而下降。
	\subsection{利润率下降是资本主义生产方式的必然结果}
	李健老师为了追求更高额的利润,势必要生产更多的香锅,并且要用机械化作业代替手工作业,结果就导致无差别的劳动减少,价格下降,利润率同时也下降。
	
	\subsection{利润率的下降伴随着利润量的增加}
	不多说
	\section{起反作用的各种原因}
	\subsection{I.劳动剥削程度的提高}
	996福报论,懂得都懂,资本家通过延长工时提高了生产无偿劳动的时间,从而提高了剩余价值率
	\subsection{II.工资被压低到劳动力的价值以下}
	我也不知道为啥无关。。。。但马克思他老人家就是这么写的啊。。。。其实最明显的例子是富士康。。。。。
	\subsection{III.不变资本各要素变得便宜}
	进货的价格下来了,自然就利润率高了嘛。。。。
	\subsection{IV.相对过剩人口}
	农村城市化导致一批人从农业生产中脱离,这些人中的大多数成为农民工,有效地降低了劳动力成本(剩余价值率),也就起到了对利润率下降的反冲。生化环材。。。。每年学的人太多但是需要的人很少。。。。。So。。。。。。
	\subsection{V.对外贸易}
	所以原装进口真的只是运费贵这么简单吗。。。。不是的。。。。只是利润率要高。。。。。。。(日本马桶盖)
	\section{规律的内部矛盾展开}
	\subsection{I.概论}
	显然,减少剥削是不可能的,这辈子都不可能的,资本家还是资本家,是要吃人的。。只不过是资本家所需要剥削的劳动少了(因为生产水平的提高)
	\subsection{II.生产扩大和价值增值之间的冲突}
	生产扩大带来的技术革新造成了价值的贬值,而为了扩大利润资本家又要使价值增值,比如说老黄和他的2080显卡,虽然技术在升级,但是为了保证利润故意限制货量,使得价格高的离谱
	\subsection{III.人口过剩时的资本过剩}
	乍看这句话(ppt)好像有点摸不到头脑,但是仔细想想便不难发现,现在的财富并不多,而是剥削无产阶级的财富太多了,而是不在无产阶级手中,用于生产的资本太多了
	\subsection{蚂蚁金服}
	好,又是“人民资本家”马先生的瓜。马先生在外滩的一通骚操作舒适迷惑,在许多事项是在国家对他睁一只眼闭一只眼的前提下试图挑战国家体系,emmmmm。。。。。不予置评。但是,这充分说明了资本主义在生产的疯狂性,为了再生产,不惜挑战国家,不惜以全国人的利益替他背锅
	\subsection{more}
	最后是ppt上没有的东西。可见,资本家为了保证利润的增加一边盲目扩大生产并进行更严重的剥削,同时伴随着大量的产品产出,而如前所述,产品要被消费才会产生剩余价值,而随着无产阶级被剥削的必要劳动时间支付不起产品时,就会导致资本市场的大崩溃,而为了扩大市场,必然就会试图去控制国家机器,并进一步控制全球经济,这就是资本主义国家的真实嘴脸。
\end{document}