\documentclass[UTF8]{ctexart}
\usepackage{geometry}
\geometry{a4paper,left=2.70cm,right=2.5cm,top=3.8cm,bottom=3.6cm}
\title{读书报告-《自卑与超越》}
\author{左熙辰-200012103}
\date{}
\begin{document}
    \maketitle
    《自卑与超越》是由奥地利心理学家,个体心理学的先驱阿德勒撰写的通俗性读物,中译本由曹晚红译,中国友谊出版公司于2017年出版。
    
    阿德勒(1870~1937),奥地利心理学家个体心理学创始人,人本主义创始人先驱,现代自我心理学之父。曾为弗洛伊德精神分析学派的核心成员之一,
    是精神分析学派中首位反对弗洛伊德的学者 1934年定居纽约,1937年赴苏格兰亚伯丁做演讲时因过度劳累,导致心脏病突发病逝。

    本文主要讲述了人为什么活着,心灵和肉体之间的关系,自卑感与优越感,家庭和学校对人的影响,而且还论及了早期记忆、梦、犯罪、爱情和婚姻等内容。
    书中着重论述了自卑感的形成以及对个人的影响,个人如何超越自卑感,如何将自卑感转变为对优越地位的追求以取得成就。

    虽然是一本通俗性读物,但其中包含着极深的哲理和颇丰的学术创见,本书修正了弗洛伊德泛性论的精神分析观,开创了精神分析学的新阶段。

    自卑感在阿德勒的理论中占有很大份量。他认为一般的自卑感是行为的原初的动力,自卑感本身是正常的,它是一个人在追求优越地位时的一种正常的心理活动,
    优越感是每个人自发追求的最终目标,它因每个人赋予生活的意义而不同。
    阿德勒提出的自卑与补偿作用,是生活中普遍存在的。许多人认为,阿德勒对心理学的贡献甚至超过了弗洛伊德。
    不仅如此,阿德勒理论的社会意义也颇为深刻,心理学家墨菲指出:“阿德勒的心理学在心理学历史中是第一个沿着我们今天应该称之为社会科学的方向发展的心理学体系”。


    本书分为两部分:认识自卑、超越自卑。认识自卑部分分别从母亲的影响,父亲的角色和责任、家庭成员关系、学校的责任课堂的合作和竞争、童年的记忆、
    早期的记忆、青春期的心理特征、青春期行为的目的、青春期的挣扎、心灵与肉体的交互作用、心灵发展的重要性进行论述;超越自卑部分则从自卑感与优越感、
    生活的意义、职业、爱情与婚姻、梦的解析、犯罪及其预防等方面进行阐述。

    本文认为,在日常生活中,人们做出一些自己明知道并不理智的行为,比如沉迷于忙碌的状态中,逃避可能出现的困难,刻意在他人面前表现自己,
    不想与任何人产生交流,对别人的优秀感到痛苦,无法忍受别人的忽视,想方设法报复他人,或是突然间大发脾气甚至是伤害他人等行为的根本原因,
    是人皆有之的自卑,而这些自卑感往往源于童年的经历:家庭的贫困、父母的忽视、他人的欺凌、生理的缺陷等等。这是因为在儿童阶段,
    人对世界的感知能力很强,但是解读能力很差。当我们感受到痛苦的时候,如果没有得到正确的引导,我们就很容易将这些痛苦当成自己的过错,
    把这些痛苦内化成我们自己的一部分,从而形成自卑。而寻找优越感,就是一种错误面对自卑感的方式:从内心寻找优越感——我只要不去努力,
    就不会暴露自己的能力不足;从外在寻找优越感——必须要比别人努力和成功,刻意引起他人注意。绝大多数不理智的行为,本质上都是通过寻找优越感来掩盖自卑的行为。


    而正确对待自卑的态度则是:认识自卑与学会合作。第一种方法是从根本上认清我们的自卑:当我们不认为自己的能力不足时,
    但如果想要正确认识自己的自卑感,首先要找到自己自卑的根源所在,还要通过学习来掌握认清它的知识,再加上这种自卑感通过长年的积累,
    已经深深地内化成了我们自身的一部分,所以仅靠我们自己是很难做到的,必须要得到他人的帮助,这也就是正确面对自卑的第二种方法——学会合作。如果我们的职业目标是为社会产生价值,
    与同伴之间更多的是合作关系而不是竞争意识,并且与爱人平等相处、相互关爱的话,我们就能真正跳出寻找优越感的陷阱,从而坦然面对自己的自卑和不足,
    因为我们明白人无完人,而我们的不足是可以通过他人的力量弥补的。

    本书中的许多地方直到现在仍值得我们借鉴。

    文中首先引出读者的深思:母亲如何给予孩子以不越位的母爱,母爱在出生之初便十分重要,母亲的第一职责就是,当孩子出生时,就让孩子感受到她是可以被依赖的;
    但我们也确实地看到了一些被宠坏的儿童:他们采用一切手段引起母亲注意,并拼命排斥与世界外沿的注意。这引发我们对母爱的深切思考。

    当一个母亲成功地和孩子建立联系后,她第二个工作就是将孩子的兴趣引导到父亲身上。而夫亲的教育方式十分重要。处罚,尤其是体罚,对孩子的伤害是很大的,最正确的教育方式就是友善的教育方式。

    作者指出,婚姻是一种合作关系,所以两个人都不应该试图驾驭对方。作为父亲,他必须证明他是妻子的好丈夫、孩子的好父亲、社会的好公民。
    他必须通过正确的方式应对生活中的三个问题:事业、友情、爱情。他必须以平等的地位与妻子合作,并照顾保护他的家庭。
    在家庭生活中女性的创造性地位是要受到尊重的,丈夫的责任不是贬低妻子的母亲角色,而是和她一起工作。
    在金钱方面,作者特别强调,即使父亲是家庭的主要经济来源,财富仍然是家庭共有的,父亲绝对不应该表现出他在施舍,别人在接受。
    
    作者的这些关于两性地位的理念,即使是处于二十一世纪的今天,也是非常先进且有教育意义的观点。夫妻双方应该是一种平等合作而非互相贬低互相利用的态度
    微博女权解决不了任何问题,只能调起两性对立。

    文中指出,父亲和自己的父母、家人相处的好,对孩子的合作能力更有利。这教育我们:要保持与父母、家人的良好关系,这对于孩子未来的发展十分重要。

    文中同样指出,一个人的记忆是他的故事,他会用这个故事不断使自己集中精力完成现实的目标让自己以成熟的姿态面对未来。记忆代表着一个人对事物的判断。

    而经验本身所留下的感受和判断,将对一生产生巨大的影响甚至是束缚。
\end{document}